\chapter{Plancherel's Theorem}
\label{chap:plancherel}
\section{Basic Properties of the Fourier Transform}
In this section we record, mostly without proofs, basic statements about the Fourier transform on
$L^1$ functions. Most of these are already formalized in mathlib.

Let $(V,\mu),(W,\nu)$ be vector spaces over $\R$ with a $\sigma$-finite measure,
$E,F$ be normed spaces over $\C$ and let $L:V\times W\to \R$, $M:E\times F\to\C$
be bilinear maps.
\begin{definition}
  \label{def:fourier-transform}
  % \uses{}
  \lean{VectorFourier.fourierIntegral, Fourier.fourierIntegral, Real.fourierIntegral,
        Real.fourierIntegralInv}
  \leanok
  Let $f\in L^1(V,E)$. Its \emph{Fourier transform} (w.r.t. $L$) is the function
  $\mathcal Ff=\widehat f:W\to E$ given by
  $$\mathcal Ff(w):=\widehat f(w):=\int_V e^{-2\pi i L(v,w)}f(v)\,d\mu.$$
  The \emph{inverse Fourier transform} (w.r.t. $L$) is similarly defined as
  $$\mathcal F^{-1}f(w)=\check f(w):=\int_Ve^{2\pi iL(v,w)}f(v)\,d\mu.$$
\end{definition}

\begin{lemma}
  \label{lem:fourier-bounded}
  \uses{def:fourier-transform}
  \lean{VectorFourier.fourierIntegral_convergent_iff,
       VectorFourier.norm_fourierIntegral_le_integral_norm}
  \leanok
  Let $f\in L^1(V,E)$. Then its Fourier transform $\widehat f$ is well-defined and bounded.
  In particular, the Fourier transform defines a map $\mathcal F:L^1(V,E)\to L^\infty(V,E)$.
\end{lemma}

From now on assume that $V$ and $W$ are equipped with second-countable topologies such that
$L$ is continuous.
\begin{lemma}
  \label{lem:fourier-cont}
  \uses{def:fourier-transform}
  \lean{VectorFourier.fourierIntegral_continuous}
  \leanok
  Let $f\in L^1(V,E)$. Then $\widehat f$ is continuous.
\end{lemma}

\begin{lemma}[Multiplication formula]
\label{lem:fourier-multiplication}
\uses{lem:fourier-cont}
\lean{VectorFourier.integral_bilin_fourierIntegral_eq_flip}
\leanok
Let $f,g\in L^1(V,E)$. Then
$$\int_WM(\widehat f(w),g(w))\,d\nu(w)=\int_VM(f(v),\widehat g(v))\,d\mu(v).$$
\end{lemma}

\begin{lemma}
  \label{lem:fourier-prop}
  \uses{def:fourier-transform}
  \lean{}
  % \leanok
Lef $f,g\in L^1(V,E)$, $t\in\R$ and $a,b\in\C$. The Fourier transform satisfies the following elementary properties:
\begin{enumerate}
  \item[(i)] $\mathcal F(af+bg)=a\mathcal Ff+b\mathcal Fg$\hfill(Linearity)
  \item[(ii)] $\mathcal F(f(x-t)) = e^{-2\pi i ty}\mathcal F f(y)$\hfill (Shifting)
  \item[(iii)] $\mathcal F(f(tx)) = \frac1{|t|}\mathcal F f(\frac yt)$\hfill (Scaling)
  \item[(iv)] $\mathcal F(\overline{f(x)}) = \overline{\mathcal Ff(-y)}$\hfill (Conjugation)
  \item[(v)] Define the convolution of $f$ and $g$ w.r.t. a bilinear map $M:E\times E\to F$ as
             $$(f\ast_Mg)(w):=\int_VM(f(v),g(w-v))\,d\mu(v).$$ Then
             $\mathcal F(f\ast_M g) =M(\mathcal Ff,\mathcal Fg)$ \hfill (Convolution)
\end{enumerate}
\end{lemma}
From now on, let $V$ be a finite-dimensional inner product space.
We denote this product as ordinary multiplication, and the induced norm as $|\cdot|.$

We now study a family of functions which is useful for later proofs.
\begin{lemma}
\label{lem:fourier-gaussian}
\uses{def:fourier-transform, lem:fourier-prop}
\lean{fourierIntegral_gaussian_innerProductSpace'}
\leanok
Let $x\in V$ and $\delta>0$. Define the \emph{modulated Gaussian}
$$u_{x,\delta}(y):V\to\C,\quad y\mapsto e^{-\pi|y|^2}e^{2\pi i x y}.$$
Its Fourier transform (w.r.t. the inner product) is given by
$$\widehat{u_{x,\delta}}(z)=\delta^{-n/2}e^{-\pi|x-y|^2/\delta}=:K_\delta(x-z).$$
\end{lemma}
\begin{proof}
  \leanok
  By choosing an orthonormal basis, wlog we may assume $V=\R^n$.
  First note $\widehat{u_{x,\delta}}(z-x)=\widehat{u_{0,\delta}}(z)$, so it is enough to consider $x=0$.
  Next, $$\widehat{u_{0,\delta}}(z)=\int_{\R^n}e^{-\pi\delta|y|^2-2\pi iyz}\,dy=
  \prod_{i=1}^n\int_{\R}e^{-\pi\delta y_i^2-2\pi iy_iz_i}dy_i\quad\text{and}\quad
  K_\delta(-z)=\prod_{i=1}^n\delta^{-1/2}e^{-\pi y_i^2/\delta},$$ hence we may assume $n=1$.
  The change of variables $w=\delta^{1/2}y+iz/\delta^{1/2}$ results in
  $$\widehat{u_{0,\delta}}(z)=\int_{\R} e^{-\pi\delta y^2-2\pi i y z}\,dy=
  \delta^{-1/2}e^{-\pi z^2/\delta}\int_{Im(w)=z/\delta^{1/2}}e^{-\pi w^2}\,dw.$$
  Contour integration along the rectangle with vertices $(\pm R,0),(\pm R,iz/\delta^{1/2})$, together
  with the bound $$\left|\int_{\pm R}^{\pm R+iz/\delta^{1/2}}e^{-\pi w^2}\,dw\right|\leq
  \frac{|z|}{\delta^{1/2}}\sup_{w\in[R,R+iz/\delta^{1/2}]}|e^{-\pi w^2}|=
  \frac{|z|}{\delta^{1/2}}e^{-\pi R^2}\xrightarrow{R\to\infty}0$$
  yields
  $$\int_{Im(w)=-z/\delta^{1/2}}e^{-\pi w^2}\,dw=
  \int_{\R} e^{-\pi w^2}\,dw=1,$$ finishing the proof.
\end{proof}

\begin{lemma}
\label{lem:weierstrass-kernel}
% \uses{}
% \lean{}
% \leanok
Let $K_\delta(v)=\delta^{-n/2}e^{-\pi|v|^2/\delta}$
as in \Cref{lem:fourier-gaussian}. This is a
\emph{good kernel}, called the \emph{Weierstrass kernel}, satisfying
$$\int_{V}K_\delta(x)\,dx= 1\quad\text{and}\quad
\int_{|x|>\eta}K_\delta(x)\,dx\xrightarrow{\delta\to0}0\quad\text{for all }\eta>0.$$
Furthermore, it satisfies the stronger bounds
$$K_\delta(x)\leq \delta^{-n/2}\quad\text{and}\quad K_\delta(x)\leq B\delta^{1/2}|x|^{-n-1}$$ for
some constant $B$ independent of $\delta$.
\end{lemma}
\begin{proof}
  % \leanok
  By choosing an orthonormal basis, wlog we may assume $V=\R^n$. Then these are all straight-forward calculations:
  $$\int_{\R^n}e^{-\pi|x|^2/\delta}\,dx=\delta^{n/2}\int_{\R^n}e^{-\pi|x|^2}\,dx=\delta^{n/2}.$$
  $$\int_{|x|>\eta}\delta^{-n/2}e^{-\pi|x|^2/\delta}\,dx=\int_{|x|>\eta/\delta^{1/2}}e^{-\pi|x|^2}\xrightarrow{\delta\to0}0.$$
  The first upper bound is trivial. For the second one, consider for $r,z\geq0$ the inequality
  $$\Gamma(r+1)=\int_0^\infty e^{-y}y^r\,dy\geq\int_z^\infty e^{-y}y^r\,dy\geq z^r\int_z^\infty e^{-y}\,dy=z^re^{-z}.$$
  Applied to $z=\pi|x|^2/\delta$ and $r=(n+1)/2$, this gives
  $$|x|^{n+1}=\delta^{(n+1)/2}\frac{|x|^{n+1}}{\delta^{(n+1)/2}}\leq
  \underbrace{\frac{\Gamma((n+3)/2)}{\pi^{(n+1)/2}}}_{=:B}\delta^{(n+1)/2}e^{\pi|x|^2/\delta},$$
  which is equivalent to the second upper bound of the lemma.
\end{proof}

The following technical theorem is used in the proofs of both the inversion formula and Plancherel's theorem.
\begin{theorem}
\label{thm:kernel-approximation}
\uses{lem:weierstrass-kernel}
\lean{}
% \leanok
Let $f:V\to E$ be integrable. Let $K_\delta$ be the Weierstrass kernel from \Cref{lem:weierstrass-kernel},
or indeed any family of functions satisfying the conditions of \Cref{lem:weierstrass-kernel}. Then
$$(K_\delta\ast f)(x):=\int_V K_\delta(y)f(x-y)\,d\mu(y)\xrightarrow{\delta\to0}f(x)$$ in the $L^1$-norm.
If $f$ is continuous, the convergence also holds pointwise.
\end{theorem}
\begin{proof}
% \leanok
Again we may assume $V=\R^n$. Consider the difference $$\Delta_\delta(x):=(K_\delta\ast f)(x)-f(x)=
\int_{\R^n}(f(x-y)-f(x))K_\delta(y)\,dy.$$ We prove $L^1$-convergence first:
Take $L^1$-norms and use Fubini's theorem to conclude
$$\|\Delta_\delta\|_1\leq\int_{\R^n}\|f(x-y)-f(x)\|_1K_\delta(y)\,dy.$$ For $\varepsilon>0$ find $\eta>0$
small enough so that $\|f(x-y)-f(x)\|_1<\varepsilon$ when $|y|\eta$. Thus
$$\|\Delta_\delta\|_1\leq\varepsilon+\int_{|y|>\eta}\|f(x-y)-f(x)\|K_\delta(y)\,dy\leq\varepsilon+2\|f\|_1\int_{\|y\|>\eta}K_\delta(y)\,dy.$$
By one of the properties in \Cref{lem:weierstrass-kernel}, we can choose $\delta$ small enough so that the second integral
is less than $\varepsilon$, which finishes the proof in this case.

Now assume that $f$ is continuous.
Let $d=\delta^{1/2}$ and shorten $g_\delta(x,y)=|f(x-y)-f(x)|K_\delta(y)$.Then
 $$|\Delta_\delta(x)|\leq\int_{0<|y|<d}g_\delta(x,y)\,dy+\sum_{k\in\mathbb N}\int_{2^kd<|y|<2^{k+1}d}g_\delta(x,y)\,dy.$$
 To bound these integrals, consider
 $$\varphi(r)=\frac1{r^n}\int_{|y|<r}|f(x-y)-f(x)|\,dy.$$
 It is easy to see that $\varphi$ is continuous, bounded, and approaches $0$ for $r\to0$, by continuity of $f$.
 Now $$\int_{0<|y|<d}g_\delta(x,y)\,dy\overset{(\ast)}\leq d^{-n}\int_{0<|y|<d}|f(x-y)-f(x)|\,dy=\varphi(d)$$ and
 $$\int_{2^kd<|y|<2^{k+1}d}g\,dy\overset{(\ast)}\leq
 \frac{Bd}{(2^kd)^{n+1}}\int_{2^kd<|y|<2^{k+1}d}|f(x-y)-f(x)|\,dy\leq2^{n-k}B\varphi(2^{k+1}d),$$
 where for the inequalities labeled $(\ast)$ we used the upper bounds from \Cref{lem:weierstrass-kernel}.
 Together, we find
 $$|\Delta_\delta(x)|\leq\varphi(d)+C\sum_{k\in\mathbb N}2^{-k}\varphi(2^{k+1}d)$$ for
 $C=\frac{2^n\Gamma((n+3)/2)}{\pi^{(n+1)/2}}$. Say $\varphi$ is bounded by $M\in\R$ and let $\varepsilon>0$.
 Take $N$ large enough such that $\sum_{k\geq N}2^{-k}<\varepsilon$. Choose $\delta$ small enough that
 $A(2^kd)<\varepsilon/N$ for all $k<N$. Then
 $$|\Delta_\delta(x)|\leq\varepsilon/N+(N-1)C\varepsilon/N+C\varepsilon M\leq \varepsilon C(M+1).$$
\end{proof}
\begin{remark}
  One can drop the continuity assumption and still get pointwise convergence almost everywhere. The proof stays the same,
  but one focuses on Lebesgue points of $f$. It takes slightly more work to argue that $\varphi$ behaves nicely,
  but the rest of the proof stays the same.
\end{remark}

\begin{theorem}[Inversion formula]
  \label{thm:fourier-inversion}
  \uses{thm:kernel-approximation, lem:fourier-multiplication}
  \lean{MeasureTheory.Integrable.fourier_inversion, Continuous.fourier_inversion}
  Let $f:V\to E$ be integrable and continuous. Assume $\widehat f$ is integrable as well. Then
  $$\mathcal F^{-1}\mathcal F f=f.$$
  \leanok
\end{theorem}
\begin{proof}
\leanok
Apply the multiplication formula \Cref{lem:fourier-multiplication} to $u_{x,\delta}$ and $f$, and conclude with
\Cref{thm:kernel-approximation}.
\end{proof}

\begin{remark}
  Note that both assumptions are necessary, since $\mathcal F^{-1}\mathcal Ff$ is continuous, and
  only defined if $\mathcal Ff$ is integrable.
\end{remark}

\begin{theorem}[Inversion formula, $L^1$-version]
  \label{thm:fourier-inversion-L1}
  \uses{thm:kernel-approximation, lem:fourier-multiplication}
  \leanok
  Let $f\in L^1(V,E)$. If $\widehat f\in L^1(V,E)$, then $\mathcal F^{-1}\mathcal Ff=f$.
\end{theorem}

\section{\texorpdfstring{Plancherel's Theorem and the Fourier Transform on $L^2$}
                        {Plancherel's Theorem and the Fourier Transform on L2}}
Again let $V$ be a finite-dimensional inner product space over $\R$ and let $E$ be a normed space over $\C$.
%TODO: The proof needs convolutions of functions into E. Is there another proof? Or assume E has products?
\begin{theorem}[Plancherel's Theorem]
  \label{thm:plancherel}
  \uses{thm:kernel-approximation, lem:fourier-multiplication, lem:fourier-gaussian, lem:fourier-prop}
  \lean{MeasureTheory.memℒp_fourierIntegral, MeasureTheory.snorm_fourierIntegral} % the full Lean name this corresponds to (can be a comma-separated list)
  \leanok % the *statement* of the lemma is formalized
   Suppose that $f : V \to E$ is in $L^1(V,E)\cap L^2(V,E)$ and let $\widehat{f}$ be the Fourier transform of $f$. Then $\widehat{f},\check{f}\in L^2(V,E)$ and
  \[\|\widehat{f}\|_{L^2} = \|f\|_{L^2}=\|\check f\|_{L^2}.\]
   Suppose that $f : V \to E$ is in $L^1(V,E)\cap L^2(V,E)$ and let $\widehat{f}$ be the Fourier transform of $f$. Then $\widehat{f},\check{f}\in L^2(V,E)$ and
  \[\|\widehat{f}\|_{L^2} = \|f\|_{L^2}=\|\check f\|_{L^2}.\]
  \end{theorem}
    \begin{proof}
    % \leanok % uncomment if the lemma has been proven
    %Let $(e_i)_{i\in I}$ be a normalized basis of E. This choice defines a conjugation on $E$: Let $E_\R$ be
    %the $\R$-span of the $e_i$, then $E\cong E_\R\otimes_\R\C$ and we can conjugate on the second factor.
    %We denote this conjugation by $e\mapsto\overline e$.
    Let $g(x)=\overline{f(-x)}$ and apply the multiplication formula \Cref{lem:fourier-multiplication}
    to $f\ast g$ and $u_{0,\delta}$: $$\int_V\widehat{f\ast g}\cdot u_{0,\delta}(x)\,dx=\int_V(f\ast g)(x)K_\delta(-x)\,dx
    \overset{\delta\to0}\to(f\ast g)(0)=\int_Vf(x)\overline{f(x)}\,dx=\| f\|_2^2$$ by \Cref{thm:kernel-approximation}.
    On the other hand, by \Cref{lem:fourier-prop} the left hand side simplifies to
    $$\int_V|\widehat f(x)|^2e^{-\delta\pi|x|^2}\,dx\xrightarrow{\delta\to0}\|\widehat f\|_2^2$$ by dominated convergence.

    Since $\check f(x)=\widehat f(-x)$, the corresponding statements for $\check f$ follow immediately from the ones for $\widehat f$.
\end{proof}

We now want to extend the Fourier transform to $L^2(V,E)$. For this, take a sequence of functions $(f_n)_n\subset L^1(V,E)\cap L^2(V,E)$
such that $f_n\xrightarrow[L^2]{}f$. Such sequences exist:
\begin{lemma}
  \label{lem:L12-dense}
  %\uses{MeasureTheory.Memℒp.exists_hasCompactSupport_snorm_sub_le}
  \lean{}
  % \leanok
  $L^1(V,E)\cap L^2(V,E)$ is dense in $L^2(V,E)$.
\end{lemma}
\begin{proof}
  %\leanok
  It is well-known that the space of compactly supported continuous functions is dense in every $L^p(V,E)$.
  Since those are contained in $L^1(V,E)\cap L^2(V,E)$, the claim follows.
\end{proof}

Let $f\in L^2(V,E)$. Plancherel's theorem lets us now approximate a potential $\widehat f$:
\begin{lemma}
  \label{lem:fourier12-cauchy}
  \uses{lem:L12-dense, thm:plancherel, lem:fourier-prop}
  \lean{}
  % \leanok
  Let $f\in L^2(V,E)$ and $(f_n)_n\subset L^1(V,E)\cap L^2(V,E)$ a sequence with $f_n\xrightarrow[L^2]{}f$. Then $(\widehat f_n)_n$ is a Cauchy sequence,
  hence converges in $L^2(V,E)$.
\end{lemma}
\begin{proof}
  % \leanok
  $$\|\widehat f_n-\widehat f_m\|_2=\|\widehat{f_n-f_m}\|_2\overset{\text{Plancherel}}=\|f_n-f_m\|_2$$ goes to $0$ for $n,m$ large, as $(f_n)_n$ is
  convergent, hence Cauchy. Since $L^2(V,E)$ is complete, $(\widehat f_n)_n$ converges.
\end{proof}
\begin{definition}
  \label{def:fourier-L2}
  \uses{lem:fourier12-cauchy}
  \lean{}
  % \leanok
  Let $f\in L^2(V,E)$ and take a sequence $(f_n)_n\subset L^1(V,E)\cap L^2(V,E)$ with $f_n\xrightarrow[L^2]{}f$. Set
  $$\mathcal Ff:=\widehat f:=\lim_{n\to\infty}\widehat{f_n},$$ the limit taken in the $L^2$-sense.
\end{definition}
\begin{lemma}
  \label{lem:fourier2-welldef}
  \uses{def:fourier-L2, lem:fourier12-cauchy, thm:plancherel}
  \lean{}
  % \leanok
  This is well-defined: By \Cref{lem:fourier12-cauchy}, the limit exists. Further it does not depend
  on the choice of sequence $(f_n)_n$. If $f\in L^1(V,E)\cap L^2(V,E)$, this definition agrees with
  the Fourier transform on $L^1(V,E)$.
\end{lemma}
\begin{proof}
  % \leanok
  Let $(g_n)_n$ be another sequence approximating $f$. Then
  $$\|\widehat f_n-\widehat g_n\|_2=\|f_n-g_n\|\leq\|f_n-f\|+\|g_n-g\|\to0.$$
  If $f\in L^1(V,E)\cap L^2(V,E)$, we can choose the constant sequence $(f_n)_n=(f)_n$.
\end{proof}

\begin{definition}
  \label{def:invFourier-L2}
  \uses{lem:fourier12-cauchy, thm:plancherel}
  \lean{}
  % \leanok
  Define analogously $\mathcal F^{-1}f:=\check f:=\lim_{n\to\infty}\check f_n$, if $f_n\xrightarrow[L^2]{}f\in L^2(V,E)$
  with $(f_n)_n\subset L^1(V,E)\cap L^2(V,E)$. By the same arguments as above, this is well-defined.
\end{definition}

\begin{corollary}
  \label{thm:fourier2-properties}
  \uses{def:fourier-L2, def:invFourier-L2, thm:plancherel, thm:fourier-inversion, lem:fourier-prop}
  \lean{}
  % \leanok
  Plancherel's Theorem, the inversion formula, and the properties of \Cref{lem:fourier-prop} hold
  for the Fourier transform on $L^2(V,E)$ as well.
\end{corollary}
\begin{proof}
  % \leanok
  All of these follow immediately from the definition and the observation, that all operations (norms, sums, conjugation, $\ldots$) are continuous.
  For example, let $f\in L^2(V,E)$ and take an approximating sequence $(f_n)_n$ as before. Then
  $$\|\widehat f\|_2=\|\lim_{n\to\infty}\widehat f_n\|_2=\|\lim_{n\to\infty}\|\widehat f_n\|_2=\lim_{n\to\infty}\|f_n\|_2
  =\|\lim_{n\to\infty}f_n\|_2=\|f\|_2.$$
\end{proof}

\begin{corollary}
  \label{thm:fourier-is-l2-linear}
  \uses{thm:fourier2-properties} % the (list of) LaTeX labels of the lemmas that are used for this result
  \lean{MeasureTheory.fourierIntegralL2} % the full Lean name this corresponds to (can be a comma-separated list)
  \leanok % the *statement* of the lemma is formalized
  The Fourier transform induces a continuous linear map $L^2(V,E) \to L^2(V,E)$.
  \end{corollary}
  \begin{proof}
    % \uses{} % Put any results used in the proof but not in the statement here
    % \leanok % uncomment if the lemma has been proven
    This follows immediately from \Cref{thm:fourier2-properties}: Linearity from the $L^2$-version of
    \Cref{lem:fourier-prop}, and continuity and well-definedness from the $L^2$-version of Plancherel's theorem.
\end{proof}